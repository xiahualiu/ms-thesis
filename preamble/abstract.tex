\chapter*{Abstract}

This thesis describes the design and development work of a marine debris cleanup system - Piranha. It is an autonomous water surface vehicle that can collect different types of garbage such as plastic bottles and bags. To begin with, Piranha is simple, with minimum moving parts including only two thrusters and a lever system. Necessary sensors like IMUs and GPS on Piranha enable positioning and navigation functions. Besides sensors, a LoRa wireless module allows the human worker to control Piranha from a distance. The simulation software for Piranha is discussed in chapter four with the theoretical equations and numerical methods. In the end, some control algorithms, including a 13 state EKF, a heading controller, and a trajectory tracking controller, are brought up to achieve autonomy at a certain level. This report reveals the development cycle of Piranha, from the very simple low-level mechanical parts to the high-level robotics system design.